% =========================================================  
% Capítulo 2 — Métodos de Busca  
% (este arquivo é incluído pelo main.tex dentro de \chapter)  
% =========================================================  
  
\section{Modelagem do Problema como Espaço de Estados}  
  
O \emph{8-puzzle} é um problema clássico de busca em Inteligência Artificial, modelado por um espaço de estados $S$, operadores $A$, estado inicial $s_0$ e um conjunto de estados objetivo $G$ \cite{russell2010artificial}. Cada estado é uma configuração do tabuleiro $3\times 3$, onde as peças são permutadas por movimentos do espaço vazio (representado por um número ou símbolo especial).  
  
Formalmente, o problema pode ser descrito como:  
\begin{itemize}  
    \item $S$: conjunto de todas as permutações possíveis das peças (incluindo o espaço vazio).  
    \item $s_0 \in S$: estado inicial fornecido.  
    \item $G \subset S$: conjunto contendo o estado objetivo (configuração ordenada).  
    \item $A(s)$: conjunto de operadores aplicáveis em $s$, correspondendo aos movimentos possíveis do espaço vazio (cima, baixo, esquerda, direita).  
    \item Função de custo $c(s, a, s')$: definida a seguir.  
\end{itemize}  
  
\begin{equation}  
    c(s, a, s') = 1  
\end{equation}  
  
\noindent  
Para cada movimento realizado (do estado $s$ para $s'$ por meio da ação $a$), atribui-se custo unitário \cite{russell2010artificial}.  
  
Nem toda permutação é solucionável. Para o 8-puzzle, um estado é solucionável se o número de inversões (pares de peças fora da ordem) é par \cite{nilsson1998}.  
  
\section{Busca de Custo Uniforme (UCS)}  
  
A \emph{Busca de Custo Uniforme} (Uniform Cost Search, UCS) é um algoritmo que expande sempre o nó de menor custo acumulado $g(n)$ a partir do estado inicial. Trata-se de um caso particular do algoritmo $A^*$ com heurística nula ($h(n) \equiv 0$).  
  
A fronteira é implementada como uma fila de prioridade ordenada por $g(n)$. O algoritmo garante encontrar o caminho de menor custo (ótimo), desde que todos os custos sejam positivos, como ocorre no 8-puzzle \cite{russell2010artificial}.  
  
\begin{equation}  
    f(n) = g(n)  
\end{equation}  
  
\noindent  
onde $g(n)$ é o custo do caminho do estado inicial até $n$.  
  
\section{Algoritmo $A^*$}  
  
O algoritmo $A^*$ é uma generalização da UCS que utiliza uma função heurística $h(n)$ para estimar o custo restante até o objetivo. A cada passo, expande-se o nó com menor valor de:  
  
\begin{equation}  
    f(n) = g(n) + h(n)  
\end{equation}  
  
\noindent  
onde:  
\begin{itemize}  
    \item $g(n)$: custo do caminho do nó inicial até $n$;  
    \item $h(n)$: estimativa (heurística) do custo de $n$ até o objetivo.  
\end{itemize}  
  
Quando $h(n)$ é \emph{admissível} (nunca superestima o custo real) e \emph{consistente} (ou monotônica), o $A^*$ é completo e ótimo \cite{russell2010artificial,nilsson1998,ertel2017}.  
  
\section{Heurísticas para o 8-puzzle}  
  
Heurísticas são funções $h: S \to \mathbb{N}$ que estimam o custo mínimo do estado corrente até o objetivo. Para o 8-puzzle, destacam-se:  
  
\begin{enumerate}[label=(\alph*)]  
    \item \textbf{Peças Fora do Lugar} (\emph{Misplaced Tiles}):  
    \begin{equation}  
        h_1(n) = \sum_{i=1}^{8} \mathbb{I}[x_i \neq x_i^*]  
    \end{equation}  
    onde $x_i$ é a posição da peça $i$ em $n$ e $x_i^*$ sua posição no objetivo.  
  
    \item \textbf{Distância de Manhattan}:  
    \begin{equation}  
        h_2(n) = \sum_{i=1}^{8} \left( |l_i - l_i^*| + |c_i - c_i^*| \right)  
    \end{equation}  
    onde $(l_i, c_i)$ é a linha e coluna da peça $i$ em $n$, e $(l_i^*, c_i^*)$ no objetivo.  
  
    \item \textbf{Heurísticas Não-Admissíveis}: São funções que podem superestimar o custo real, podendo tornar o algoritmo não ótimo. Exemplo: $h_3(n) = 2 \cdot h_2(n)$.  
\end{enumerate}  
  
Tanto $h_1$ quanto $h_2$ são heurísticas admissíveis e consistentes \cite{russell2010artificial}.  
  
\section{Propriedades Teóricas dos Métodos de Busca}  
  
\subsection{Admissibilidade e Consistência}  
  
Uma heurística $h(n)$ é \textbf{admissível} se, para todo $n$,  
\begin{equation}  
    0 \leq h(n) \leq h^*(n)  
\end{equation}  
onde $h^*(n)$ é o custo real mínimo de $n$ ao objetivo.  
  
Ela é \textbf{consistente} (ou monotônica) se, para todo par de estados $n$ e $n'$ tal que $n'$ é sucessor de $n$:  
\begin{equation}  
    h(n) \leq c(n, n') + h(n')  
\end{equation}  
Se $h$ é consistente, $A^*$ nunca expande um mesmo nó mais de uma vez.  
  
\subsection{Complexidade}  
  
A complexidade temporal de UCS e $A^*$ depende do fator de ramificação $b$ e da profundidade da solução $d$:  
\begin{itemize}  
    \item \textbf{UCS}: $O(b^{C^*/\epsilon})$, onde $C^*$ é o custo da solução ótima e $\epsilon$ é o menor custo de ação.  
    \item \textbf{$A^*$}: no pior caso, igual à busca em largura, mas pode ser substancialmente menor com heurística informada.  
\end{itemize}  
  
\subsection{Ótimo e Completude}  
  
\begin{itemize}  
    \item \textbf{Busca de Custo Uniforme}: ótima e completa para custos positivos.  
    \item \textbf{$A^*$}: ótima e completa se $h$ for admissível.  
\end{itemize}  
  
\section{Resumo}  
  
Neste capítulo, apresentaram-se os fundamentos dos métodos de busca aplicados ao 8-puzzle, incluindo a modelagem do espaço de estados, o funcionamento dos algoritmos UCS e $A^*$, as principais heurísticas utilizadas e suas propriedades teóricas. Para detalhes de implementação e experimentos, ver Capítulo~3.  
  
% ---------------------------------------------------------  