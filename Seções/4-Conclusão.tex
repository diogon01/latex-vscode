% =========================================================  
% Capítulo 4 — Conclusão  
% =========================================================  
  
Neste trabalho, realizou-se uma análise e implementação dos principais métodos de busca aplicados ao problema clássico do 8-puzzle, abrangendo desde a fundamentação teórica até a validação prática por meio de experimentos reprodutíveis.  
  
No Capítulo~1, foram apresentados os conceitos fundamentais de espaços de estados, operadores, funções de custo e heurísticas, ressaltando a importância da modelagem precisa para a eficácia dos algoritmos de busca. No Capítulo~2, detalhou-se o funcionamento dos algoritmos de Busca de Custo Uniforme (UCS) e $A^*$, com ênfase nas condições de admissibilidade e consistência das heurísticas.  
  
O desenvolvimento prático, abordado no Capítulo~3, evidenciou a eficiência dos métodos implementados, destacando o uso de tuplas para representação dos estados, a verificação de solubilidade das instâncias, a geração de tabuleiros aleatórios e a avaliação experimental dos algoritmos. Os resultados obtidos demonstraram o impacto significativo do uso de heurísticas informadas, como a distância de Manhattan, principalmente na redução do número de nós expandidos e na maior eficiência do algoritmo $A^*$ em comparação ao UCS.  
  
Apesar dos avanços alcançados, observou-se que o crescimento exponencial do espaço de estados para instâncias mais complexas limita a escalabilidade dos métodos tradicionais, indicando a necessidade de abordagens mais sofisticadas, como técnicas de poda, otimização adicional ou uso de heurísticas aprimoradas.  
  
Como trabalho futuro, recomenda-se a extensão dos métodos implementados para variantes do problema, como o 15-puzzle, bem como a investigação de heurísticas mais avançadas ou de metaheurísticas. A paralelização dos algoritmos e a análise de desempenho em diferentes ambientes computacionais também se mostram promissoras para aprofundar a pesquisa.  
  
Em síntese, conclui-se que a escolha de heurísticas admissíveis e uma modelagem eficiente do espaço de estados são fatores essenciais para o desenvolvimento de algoritmos de busca otimizados em problemas combinatórios, como o 8-puzzle, contribuindo para a aproximação entre teoria e prática no campo dos sistemas inteligentes.  
  
% ---------------------------------------------------------  