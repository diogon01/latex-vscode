% =========================================================  
% Capítulo 4 — Conclusão  
% =========================================================  
  
Neste trabalho, realizou-se uma análise e implementação dos principais métodos de busca aplicados ao problema clássico do 8-puzzle, abordando desde a fundamentação teórica até a validação prática por meio de experimentos.  
  
No Capítulo~1, foram apresentados os conceitos fundamentais de espaços de estados, operadores, funções de custo e heurísticas, destacando a importância da modelagem adequada para o sucesso dos algoritmos de busca. No Capítulo~2, detalhou-se o funcionamento dos algoritmos de Busca de Custo Uniforme (UCS) e $A^*$, com ênfase nas condições de admissibilidade e consistência das heurísticas.  
  
O desenvolvimento prático, abordado no Capítulo~3, evidenciou a eficiência dos métodos implementados, destacando o uso de tuplas para representação dos estados, a verificação de solubilidade das instâncias, a geração de tabuleiros aleatórios e a avaliação experimental dos algoritmos. Os resultados obtidos demonstraram o impacto positivo do uso de heurísticas informadas, como a distância de Manhattan, especialmente na redução do número de nós expandidos e na eficiência do algoritmo $A^*$ em comparação ao UCS.  
  
Apesar dos avanços, observou-se o crescimento exponencial do espaço de estados para instâncias mais complexas, o que pode demandar abordagens mais sofisticadas, como técnicas de poda ou otimização adicional.  
  
Como trabalho futuro, recomenda-se a aplicação dos métodos implementados a variantes do problema, como o 15-puzzle, além da investigação de heurísticas mais avançadas ou metaheurísticas. A paralelização dos algoritmos e a análise de desempenho em diferentes ambientes computacionais também podem contribuir para o aprofundamento da pesquisa.  
  
Conclui-se, portanto, que o uso de heurísticas admissíveis e uma modelagem eficiente do espaço de estados são essenciais para o desenvolvimento de algoritmos de busca otimizados em problemas combinatórios, como o 8-puzzle.  
  
% ---------------------------------------------------------  