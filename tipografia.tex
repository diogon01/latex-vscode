% ==============================
% Tipografia / Estilo do documento
% ==============================

% Cores e tabelas coloridas (necessário para \rowcolor e \textcolor)
\usepackage[table,xcdraw]{xcolor}
\usepackage{colortbl}

% Títulos (cap/section)
\usepackage[explicit]{titlesec}

% Sumário estilizado (sem KOMA/tocbasic para evitar conflito)
\usepackage[dotinlabels]{titletoc}

% --------- Cores de conveniência ----------
\newcommand{\cor}{black}              % cor “tema”
\newcommand{\cTT}{\rowcolor{\cor!10}} % header de tabela
\newcommand{\cLT}{\rowcolor{\cor!5}}  % linha de destaque

% --------- Marcadores quadrados ----------
\newcommand\MySquare{%
  \leavevmode\hbox to 1.2ex{\hss\vrule height .9ex width .7ex depth -.2ex\hss}}
\newcommand\MySecSquare{%
  \leavevmode\hbox to 1.2ex{\hss\vrule height 1.3ex width 1.1ex depth -.2ex\hss}}
\renewcommand\labelitemi{\MySquare}

% --------- Formatação de CAPÍTULO ----------
\titleformat{\chapter}[display]
  {\normalfont\sffamily\bfseries\LARGE\color{black}}
  {\renewcommand{\thechapter}{\arabic{chapter}}\hspace*{0.5em}%
   \colorbox{black}{\parbox[c][1.2cm][c]{1cm}{\centering\textcolor{white}{\Huge\thechapter}}}}
  {-0.5ex}
  {\titlerule\vspace{.7ex}\filleft\MakeUppercase{#1}}
  [\vspace{.2ex}\titlerule]

\titlespacing*{\chapter}{0pt}{48pt}{72pt}

% --------- Formatação de SECTION/SUBSECTION ----------
\titleformat{\section}
  {\normalfont\Large\bfseries\color{black}}
  {\MySecSquare\ \thesection}{1em}{#1}
\titleformat{name=\section,numberless}
  {\normalfont\Large\bfseries\color{black}}
  {\MySecSquare}{1em}{#1}

\titleformat{\subsection}
  {\normalfont\large\bfseries\color{black}}
  {\MySecSquare\ \thesubsection}{1em}{#1}
\titleformat{name=\subsection,numberless}
  {\normalfont\large\bfseries\color{black}}
  {\MySecSquare}{1em}{#1}

% --------- Sumário (ToC) ----------
\titlecontents{chapter}
  [3.8em]{}
  {\sffamily\bfseries\contentslabel{3.1em}}
  {}
  {\titlerule*[1pc]{.}\sffamily\bfseries\contentspage}

\titlecontents{section}
  [3.8em]{}
  {\sffamily\bfseries\contentslabel{3.1em}}
  {}
  {\titlerule*[1pc]{.}\sffamily\bfseries\contentspage}

\titlecontents{subsection}
  [3.8em]{}
  {\sffamily\contentslabel{3.45em}}
  {}
  {\titlerule*[1pc]{.}\sffamily\contentspage}

% ==============================
% Cabeçalho / Rodapé (ABNT-like)
% ==============================
\usepackage{fancyhdr}
\setlength{\headheight}{14pt} % evita warning de headheight
\fancyhf{}                    % limpa
% número no canto superior externo; título no interno
\fancyhead[LE,RO]{\thepage}
\fancyhead[LO]{\small\rightmark} % páginas ímpares: seção
\fancyhead[RE]{\small\leftmark}  % páginas pares: capítulo
\renewcommand{\headrulewidth}{0.4pt}
\renewcommand{\footrulewidth}{0pt}
\pagestyle{fancy}

% ==============================
% Legendas (captions)
% ==============================
\usepackage{caption}
% Estilo global (aplica a figura e tabela)
\captionsetup{
  font=small,
  labelfont={bf,color=\cor},
  labelsep=colon,
  justification=raggedright,
  singlelinecheck=false
}
% Se quiser por-tipo (descomente e comente o bloco global acima)
% \captionsetup[figure]{font=small, labelfont={bf,color=\cor}, labelsep=colon}
% \captionsetup[table]{ font=small, labelfont={bf,color=\cor}, labelsep=colon}

% Português nos rótulos
\renewcommand{\figurename}{Figura}
\renewcommand{\tablename}{Tabela}
