 %%%%%%%%%% Tipografia dos Capitulos %%%%%%%%%%%%%%%%%%%%%%%%%%%%%%%%%%%%%%%%%%%%%%%%%%%%%%%%%%%%%
		\usepackage[explicit]{titlesec}
		\usepackage[dotinlabels]{titletoc}
		
        \newcommand{\cor}{black} % Definição do novo comando para a cor
        \newcommand{\cTT}{\rowcolor{\cor!10}} % Título tabela
        \newcommand{\cLT}{\rowcolor{\cor!5}} % Título tabela
        %\definecolor{azule}{RGB}{11,131,144}
        %\newcommand{\cor}{azule} % Definição do novo comando para a cor
        %\newcommand{\cTT}{\rowcolor{\cor!10}} % Título tabela
        %\newcommand{\cLT}{\rowcolor{\cor!5}} % Título tabela
		
		% % chapter tiltes formatting
		% \titleformat{\chapter}[display]
		% 	{\normalfont\sffamily\bfseries\LARGE\color{black}}
		% 	{\renewcommand{\thechapter}{\arabic{chapter}}\hspace*{0.5em}\colorbox{black}{%
		% 	%{\renewcommand{\thechapter}{\Roman{chapter}}\hspace*{0.5em}\colorbox{black}{%
		% 		\parbox[c][1.2cm][c]{1cm}{% Altura e largura do quadrado do numero do capitulo
		% 			\centering\textcolor{white}  {\Huge\thechapter}}}}
		% 	{-0.5ex} % Distância do quadrado do número do capítulo ate o titulo do capitulo
		% 	{\titlerule\vspace{.7ex}\filleft\MakeUppercase{#1}}
		% 	[\vspace{.2ex}\titlerule]
		% % chapter tiltes spacing
		% \titlespacing*{\chapter}{0pt}{50pt}{80pt}

		% % section tiltes formatting
		% \titleformat{\section}
		% 	{\normalfont\Large\bfseries\color{black}}{\MySecSquare\ \thesection}{1em}{#1}
		% \titleformat{name=\section,numberless}
		% 	{\normalfont\Large\bfseries\color{black}}{\MySecSquare}{1em}{#1}
			
		% % subsection tiltes formatting
		% \titleformat{\subsection}
		% 	{\normalfont\large\bfseries\color{black}}{\MySecSquare\ \thesubsection}{1em}{#1}
		% \titleformat{name=\subsection,numberless}
		% 	{\normalfont\large\bfseries\color{black}}{\MySecSquare}{1em}{#1}

		% formatting for chapter entries in ToC  
		\titlecontents{chapter}
			[3.8em]{ } % 1.5
			{\sffamily\bfseries\contentslabel{3.1em}} % 1.5
			{\hspace*{0em}} % -1.5
			{\titlerule*[1pc]{.}\sffamily\bfseries\contentspage}

 
		% formatting for section entries in ToC  
		\titlecontents{section}
			[3.8em]{}
			{\sffamily\bfseries\contentslabel{3.1em}} % 2.3
			{\hspace*{0em}} % -2.3
			{\titlerule*[1pc]{.}\sffamily\bfseries\contentspage}
			
		% formatting for subsection entries in ToC  
		\titlecontents{subsection}
			[3.8em]{} % 6.9
			{\sffamily\contentslabel{3.45em}} % 3.1
			{\hspace*{0em}} % -3.1
			{\titlerule*[1pC]{.}\sffamily\contentspage}

		% Square to be used in itemize
		\newcommand\MySquare{%
			\leavevmode\hbox to 1.2ex{\hss\vrule height .9ex width .7ex depth -.2ex\hss}}
		% Square to be used in section titles
		\newcommand\MySecSquare{%
			\leavevmode\hbox to 1.2ex{\hss\vrule height 1.3ex width 1.1ex depth -.2ex\hss}}
 
		% First level of itemize uses a square
		\renewcommand\labelitemi{\MySquare}


		
	%%%%%%%%%%%%%%%%%%%%%%%%%%%%%%%%%%%%%%%%%%%%%%%%%%%%%%%%%%%%%%%%%%%%
	\usepackage{blindtext}

\makeatletter
\if@twoside
  \def\ps@headings{%
      %\let\@oddfoot\@empty\let\@evenfoot\@empty       % No footer
      \def\@evenhead{\footnotesize\thepage\hfil\leftmark}%   % Header on even pages small
      \def\@oddhead{\footnotesize\rightmark\hfil\thepage}%   % Header on  odd pages small
      \let\@mkboth\markboth
    \def\chaptermark##1{%
      \markboth{%\scshape%                             % Chapters in Header capitalized
        \ifnum \c@secnumdepth >\m@ne
            \@chapapp\ \thechapter. \ %
        \fi
        ##1}{}}%
    \def\sectionmark##1{%
      \markright{%\scshape%                            % Sections in Header capitalized
        \ifnum \c@secnumdepth >\z@
          \thesection. \ %
        \fi
        ##1}}
    \def\subsectionmark##1{%
      \markright{%\scshape%                            % Sections in Header capitalized
        \ifnum \c@secnumdepth >\z@
          \thesection. \ %
        \fi
        ##1}}}


\makeatother
\pagestyle{headings}


%%% CAPTION (para ajustar legenda das figuras e tabelas) %%%
\usepackage{caption}

\captionsetup[figure]{
  labelfont={bf,color=\cor}, % "Figura 1:" em negrito e azul
  textfont=normalfont           % Título da legenda em fonte normal
}
\captionsetup[table]{
  labelfont={bf,color=\cor}, % "Tabela 1:" em negrito e azul
  textfont=normalfont           % Título da legenda em fonte normal
}

%%% LISTAS DE FIGURAS E TABELAS %%%
    \usepackage{tocbasic}

% Prefixo colorido e em negrito
    \newcommand\entrynumberwithprefix[2]{\textbf{\textcolor{\cor}{#1~#2:}}\hfill}

% FIGURAS
    \DeclareTOCStyleEntry[  entrynumberformat=\entrynumberwithprefix{\figurename},
    dynnumwidth,
    numsep=1em
    ]{tocline}{figure}

% TABELAS
    \DeclareTOCStyleEntry[    entrynumberformat=\entrynumberwithprefix{\tablename},
    dynnumwidth,
    numsep=1em
    ]{tocline}{table}
%%% TÍTULOS das legendas: “Figura” e “Tabela” %%%
    \renewcommand{\figurename}{Figura}
    \renewcommand{\tablename}{Tabela}

%%% Posição do cabeçalho
\usepackage{fancyhdr}
\pagestyle{fancy}
\fancyhf{} % Limpa todos os campos do cabeçalho/rodapé

% Cabeçalho:
\fancyhead[C]{\small \leftmark}  % Centralizado
%\fancyhead[LE]{\small \leftmark}  % Left Even: página par (esquerda)
%\fancyhead[RO]{\small \leftmark}  % Right Odd: página ímpar (direita)
% Rodapé com numeração no centro, por exemplo:
\fancyfoot[C]{\thepage}

