\documentclass[a4paper,12pt,openany,twoside]{book}

% ========================
% Pacotes e estilo
% ========================
% ===========================
% pacotes.tex  (ABNT - autor-data)
% ===========================

% --- Codificação e idioma ---
\usepackage[utf8]{inputenc}   % pdflatex
\usepackage[T1]{fontenc}
\usepackage[brazil]{babel}
\usepackage{lmodern}

% --- Margens ABNT (3 cm esq/topo, 2 cm dir/base) ---
\usepackage[a4paper,left=3cm,right=2cm,top=3cm,bottom=2cm]{geometry}

% --- Tipografia e espaçamento ---
\usepackage{microtype}   % melhor quebra/justificação
\usepackage{setspace}    % \OnehalfSpacing no miolo
\usepackage{ragged2e}    % \justifying para garantir texto justificado
\usepackage{indentfirst} % 1º parágrafo de cada seção com recuo

% --- Matemática básica ---
\usepackage{amsmath,amssymb}

% --- Figuras e tabelas ---
\usepackage{graphicx}
\usepackage[table,xcdraw]{xcolor} % cores para tabela e \rowcolor
\usepackage{colortbl}
\usepackage{array,booktabs,tabularx,longtable}
\usepackage{float}        % [H] se necessário

% --- Legendas (estilo global; títulos PT-BR vêm em tipografia.tex se preferir) ---
\usepackage{caption}
\captionsetup{
  font=small,
  labelfont=bf,
  labelsep=colon,
  justification=raggedright,
  singlelinecheck=false
}

% --- URLs e quebras de linha longas ---
\usepackage{xurl}     % permite quebra de URLs
\urlstyle{same}

% --- Hiperlinks (cores discretas, padrão acadêmico) ---
\usepackage[hidelinks]{hyperref}
% Se preferir colorido discreto, troque por:
% \usepackage[colorlinks=true,linkcolor=black,citecolor=black,urlcolor=black]{hyperref}

% --- Citações ABNT (autor-data) + estilo BibTeX ---
\usepackage[alf]{abntex2cite}
% No main.tex, use: \bibliographystyle{abntex2-alf} \bibliography{Bibliografia}

% --- Código fonte (minted v3) ---
% Requer: -shell-escape e python+pygments+latexminted instalados
\usepackage[newfloat]{minted}
\setminted{
  frame=lines,
  linenos,
  fontsize=\footnotesize,
  breaklines,
  autogobble
}
\usemintedstyle{friendly}

% --- Índice remissivo (se usar \printindex) ---
\usepackage{makeidx}
\makeindex

% --- Listas mais controladas (opcional) ---
\usepackage{enumitem}
\setlist{itemsep=0pt,topsep=4pt}

% --- Evita viúvas/órfãs muito feias (opcional) ---
\clubpenalty=10000
\widowpenalty=10000
\displaywidowpenalty=10000
     % (ABNT autor–data, geometry, minted, etc.)
% ==============================
% Tipografia / Estilo do documento
% ==============================

% Cores e tabelas coloridas (necessário para \rowcolor e \textcolor)
\usepackage[table,xcdraw]{xcolor}
\usepackage{colortbl}

% Títulos (cap/section)
\usepackage[explicit]{titlesec}

% Sumário estilizado (sem KOMA/tocbasic para evitar conflito)
\usepackage[dotinlabels]{titletoc}

% --------- Cores de conveniência ----------
\newcommand{\cor}{black}              % cor “tema”
\newcommand{\cTT}{\rowcolor{\cor!10}} % header de tabela
\newcommand{\cLT}{\rowcolor{\cor!5}}  % linha de destaque

% --------- Marcadores quadrados ----------
\newcommand\MySquare{%
  \leavevmode\hbox to 1.2ex{\hss\vrule height .9ex width .7ex depth -.2ex\hss}}
\newcommand\MySecSquare{%
  \leavevmode\hbox to 1.2ex{\hss\vrule height 1.3ex width 1.1ex depth -.2ex\hss}}
\renewcommand\labelitemi{\MySquare}

% --------- Formatação de CAPÍTULO ----------
\titleformat{\chapter}[display]
  {\normalfont\sffamily\bfseries\LARGE\color{black}}
  {\renewcommand{\thechapter}{\arabic{chapter}}\hspace*{0.5em}%
   \colorbox{black}{\parbox[c][1.2cm][c]{1cm}{\centering\textcolor{white}{\Huge\thechapter}}}}
  {-0.5ex}
  {\titlerule\vspace{.7ex}\filleft\MakeUppercase{#1}}
  [\vspace{.2ex}\titlerule]

\titlespacing*{\chapter}{0pt}{48pt}{72pt}

% --------- Formatação de SECTION/SUBSECTION ----------
\titleformat{\section}
  {\normalfont\Large\bfseries\color{black}}
  {\MySecSquare\ \thesection}{1em}{#1}
\titleformat{name=\section,numberless}
  {\normalfont\Large\bfseries\color{black}}
  {\MySecSquare}{1em}{#1}

\titleformat{\subsection}
  {\normalfont\large\bfseries\color{black}}
  {\MySecSquare\ \thesubsection}{1em}{#1}
\titleformat{name=\subsection,numberless}
  {\normalfont\large\bfseries\color{black}}
  {\MySecSquare}{1em}{#1}

% --------- Sumário (ToC) ----------
\titlecontents{chapter}
  [3.8em]{}
  {\sffamily\bfseries\contentslabel{3.1em}}
  {}
  {\titlerule*[1pc]{.}\sffamily\bfseries\contentspage}

\titlecontents{section}
  [3.8em]{}
  {\sffamily\bfseries\contentslabel{3.1em}}
  {}
  {\titlerule*[1pc]{.}\sffamily\bfseries\contentspage}

\titlecontents{subsection}
  [3.8em]{}
  {\sffamily\contentslabel{3.45em}}
  {}
  {\titlerule*[1pc]{.}\sffamily\contentspage}

% ==============================
% Cabeçalho / Rodapé (ABNT-like)
% ==============================
\usepackage{fancyhdr}
\setlength{\headheight}{14pt} % evita warning de headheight
\fancyhf{}                    % limpa
% número no canto superior externo; título no interno
\fancyhead[LE,RO]{\thepage}
\fancyhead[LO]{\small\rightmark} % páginas ímpares: seção
\fancyhead[RE]{\small\leftmark}  % páginas pares: capítulo
\renewcommand{\headrulewidth}{0.4pt}
\renewcommand{\footrulewidth}{0pt}
\pagestyle{fancy}

% ==============================
% Legendas (captions)
% ==============================
\usepackage{caption}
% Estilo global (aplica a figura e tabela)
\captionsetup{
  font=small,
  labelfont={bf,color=\cor},
  labelsep=colon,
  justification=raggedright,
  singlelinecheck=false
}
% Se quiser por-tipo (descomente e comente o bloco global acima)
% \captionsetup[figure]{font=small, labelfont={bf,color=\cor}, labelsep=colon}
% \captionsetup[table]{ font=small, labelfont={bf,color=\cor}, labelsep=colon}

% Português nos rótulos
\renewcommand{\figurename}{Figura}
\renewcommand{\tablename}{Tabela}
  % (títulos, cabeçalho/rodapé, captions PT-BR)

% Recuo e espaçamento ABNT (reforço aqui para ficar explícito)
\setlength{\parindent}{1.25cm}
\setlength{\parskip}{0pt}

\begin{document}

% ========================
% Pré-texto (simples e romano)
% ========================
\singlespacing
\pagenumbering{roman}

% Capa
\begin{titlepage}
\begin{center}
\textbf{UNIVERSIDADE FEDERAL DE SANTA CATARINA}\\
Departamento de Informática e Estatística - INE\\
Sistemas de Informação\\

\vspace{3cm}

INE5633 - Sistemas Inteligentes\\
{\footnotesize{Disciplina}}\\
\vspace{1cm}

Elder Rizzon Santos\\
{\footnotesize{Professor}}

\vspace{4cm}

{\large{\textbf{Trabalho sobre Métodos de busca (2025/2)}}\\ \footnotesize{Atividade Prática 1}}


\vspace{6cm}
Bruno Rafael Leal Machado\\
Diogo Henrique Fragoso de Oliveira\\
José Antonio de Oliveira\\
\footnotesize{Alunos}

\vfill

\today  

\end{center}

\end{titlepage}

% Capa

% Liste apenas o que realmente usar
% \listoftables
\listoffigures
\newpage

\tableofcontents
\newpage

% ========================
% Texto principal (1,5 e arábico)
% ========================
\clearpage
\pagenumbering{arabic}
\onehalfspacing      % <<< corrigido (setspace)
\justifying          % requer \usepackage{ragged2e}

% --------- Capítulo 1: Introdução ---------  
\chapter{Introdução}\label{cap:introducao}  
% =========================
% Seções/1-Introdução.tex
% (não usar \section{Introdução} aqui;
%  o capítulo é aberto no main.tex)
% =========================

A resolução de problemas por \emph{busca} ocupa posição central na Inteligência Artificial (IA) simbólica: modela a 
navegação em um \emph{espaço de estados} por meio de operadores, avaliando custos e selecionando expansões 
segundo políticas informadas ou não informadas, como discutem \citeonline{russell2010artificial} e 
\citeonline{nilsson1998}. O interesse central está em compreender como diferentes algoritmos percorrem esse espaço, 
quais estruturas de dados utilizam e de que modo heurísticas afetam o desempenho.

Para fins experimentais, optou-se pelo \emph{8-puzzle} apenas como domínio de teste. Trata-se de um tabuleiro 
$3\times 3$ com oito peças móveis e um espaço vazio, no qual o objetivo é atingir uma configuração final a partir 
de um arranjo inicial. A escolha desse problema não se deve à sua complexidade prática, mas sim ao fato de ser 
computacionalmente leve, permitindo implementar e comparar variações de algoritmos de busca de forma controlada e 
reprodutível. Assim, o foco da análise permanece nos métodos de busca --- custo uniforme e versões do algoritmo $A^*$ 
com heurísticas de diferentes níveis de admissibilidade --- e não no quebra-cabeça em si. Nesse sentido, o 8-puzzle 
funciona como um \emph{laboratório didático} que viabiliza a avaliação sistemática de desempenho.


No contexto da disciplina INE5633---\emph{Sistemas Inteligentes} (UFSC), esta Atividade Prática 1 (AP1) utiliza o 8-puzzle como laboratório para implementar e analisar o algoritmo A* e variações, em alinhamento ao conteúdo de \emph{raciocínio e resolução de problemas} e à ênfase em técnicas de procura e informação heurística; a proposta didática privilegia implementação e análise prática, em consonância com a bibliografia básica adotada \cite{russell2010artificial,luger2009artificial}.

\section{Escopo desta Atividade Prática teste}
Serão estudadas quatro variantes: (i) busca de custo uniforme (sem heurística), (ii) A* com heurística \emph{não admissível}, (iii) A* com heurística admissível simples e (iv) A* com a heurística admissível mais precisa desenvolvida pela equipe. A comparação considerará: total de nós visitados, comprimento do caminho-solução, maior tamanho da fronteira (abertos), tempo de execução e um arquivo \texttt{.txt}/\texttt{.json} com fronteira e visitados ao término. Esses indicadores permitem discutir \emph{admissibilidade} e \emph{consistência} das heurísticas no A*, além de seus efeitos de eficiência \cite{russell2010artificial,luger2009artificial}.

\section{Objetivos formativos}
Consolidar, por meio de implementação e experimentação reprodutível, a ponte entre teoria e prática em busca. Em particular, pretende-se:
\begin{enumerate}[label=(\alph*),leftmargin=*,itemsep=0pt,topsep=2pt]
  \item formalizar o problema como \emph{espaço de estados} (definição de estados, operadores, teste de objetivo e função de custo);
  \item projetar e gerir a \emph{fronteira} com checagem de dominância e de estados repetidos (políticas de inserção/remoção e estrutura de dados apropriada);
  \item definir e justificar heurísticas (admissíveis e não admissíveis), discutindo propriedades como admissibilidade e consistência;
  \item analisar comparativamente o desempenho das variantes (UCS e A*), com métricas reprodutíveis e interpretação crítica.
\end{enumerate}
A fundamentação teórica apoia-se em \citeonline{russell2010artificial,nilsson1998} e obras complementares como \citeonline{ertel2017}.


\section{Contribuições esperadas}
O relatório apresentará:
\begin{enumerate}[label=(\roman*),leftmargin=*,itemsep=0pt,topsep=2pt]
  \item a modelagem do 8-puzzle e as estruturas de dados utilizadas;
  \item o A* e o UCS com suas políticas de fronteira e critérios de expansão;
  \item o desenho das heurísticas, com justificativa matemática e discussão sobre admissibilidade/consistência;
  \item a avaliação experimental (casos fáceis, médios e difíceis), com comparação de nós visitados, comprimento do caminho-solução, maior tamanho da fronteira, tempo de execução e arquivo \texttt{.txt}/\texttt{.json} contendo fronteira e visitados ao término.
\end{enumerate}
A análise será fundamentada em literatura clássica \cite{russell2010artificial,luger2009artificial,nilsson1998,ertel2017} e será reprodutível a partir dos artefatos entregues (código e logs).

  
  
% --------- Capítulo 2: Métodos de Busca ---------  
\chapter{Métodos de Busca}\label{cap:metodos}  
% =========================================================  
% Capítulo 2 — Métodos de Busca  
% (este arquivo é incluído pelo main.tex dentro de \chapter)  
% =========================================================  
  
\section{Modelagem do Problema como Espaço de Estados}  
  
O \emph{8-puzzle} é um problema clássico de busca em Inteligência Artificial, modelado por um espaço de estados $S$, operadores $A$, estado inicial $s_0$ e um conjunto de estados objetivo $G$ \cite{russell2010artificial}. Cada estado é uma configuração do tabuleiro $3\times 3$, onde as peças são permutadas por movimentos do espaço vazio (representado por um número ou símbolo especial).  
  
Formalmente, o problema pode ser descrito como:  
\begin{itemize}  
    \item $S$: conjunto de todas as permutações possíveis das peças (incluindo o espaço vazio).  
    \item $s_0 \in S$: estado inicial fornecido.  
    \item $G \subset S$: conjunto contendo o estado objetivo (configuração ordenada).  
    \item $A(s)$: conjunto de operadores aplicáveis em $s$, correspondendo aos movimentos possíveis do espaço vazio (cima, baixo, esquerda, direita).  
    \item Função de custo $c(s, a, s')$: definida a seguir.  
\end{itemize}  
  
\begin{equation}  
    c(s, a, s') = 1  
\end{equation}  
  
\noindent  
Para cada movimento realizado (do estado $s$ para $s'$ por meio da ação $a$), atribui-se custo unitário \cite{russell2010artificial}.  
  
Nem toda permutação é solucionável. Para o 8-puzzle, um estado é solucionável se o número de inversões (pares de peças fora da ordem) é par \cite{nilsson1998}.  
  
\section{Busca de Custo Uniforme (UCS)}  
  
A \emph{Busca de Custo Uniforme} (Uniform Cost Search, UCS) é um algoritmo que expande sempre o nó de menor custo acumulado $g(n)$ a partir do estado inicial. Trata-se de um caso particular do algoritmo $A^*$ com heurística nula ($h(n) \equiv 0$).  
  
A fronteira é implementada como uma fila de prioridade ordenada por $g(n)$. O algoritmo garante encontrar o caminho de menor custo (ótimo), desde que todos os custos sejam positivos, como ocorre no 8-puzzle \cite{russell2010artificial}.  
  
\begin{equation}  
    f(n) = g(n)  
\end{equation}  
  
\noindent  
onde $g(n)$ é o custo do caminho do estado inicial até $n$.  
  
\section{Algoritmo $A^*$}  
  
O algoritmo $A^*$ é uma generalização da UCS que utiliza uma função heurística $h(n)$ para estimar o custo restante até o objetivo. A cada passo, expande-se o nó com menor valor de:  
  
\begin{equation}  
    f(n) = g(n) + h(n)  
\end{equation}  
  
\noindent  
onde:  
\begin{itemize}  
    \item $g(n)$: custo do caminho do nó inicial até $n$;  
    \item $h(n)$: estimativa (heurística) do custo de $n$ até o objetivo.  
\end{itemize}  
  
Quando $h(n)$ é \emph{admissível} (nunca superestima o custo real) e \emph{consistente} (ou monotônica), o $A^*$ é completo e ótimo \cite{russell2010artificial,nilsson1998,ertel2017}.  
  
\section{Heurísticas para o 8-puzzle}  
  
Heurísticas são funções $h: S \to \mathbb{N}$ que estimam o custo mínimo do estado corrente até o objetivo. Para o 8-puzzle, destacam-se:  
  
\begin{enumerate}[label=(\alph*)]  
    \item \textbf{Peças Fora do Lugar} (\emph{Misplaced Tiles}):  
    \begin{equation}  
        h_1(n) = \sum_{i=1}^{8} \mathbb{I}[x_i \neq x_i^*]  
    \end{equation}  
    onde $x_i$ é a posição da peça $i$ em $n$ e $x_i^*$ sua posição no objetivo.  
  
    \item \textbf{Distância de Manhattan}:  
    \begin{equation}  
        h_2(n) = \sum_{i=1}^{8} \left( |l_i - l_i^*| + |c_i - c_i^*| \right)  
    \end{equation}  
    onde $(l_i, c_i)$ é a linha e coluna da peça $i$ em $n$, e $(l_i^*, c_i^*)$ no objetivo.  
  
    \item \textbf{Heurísticas Não-Admissíveis}: São funções que podem superestimar o custo real, podendo tornar o algoritmo não ótimo. Exemplo: $h_3(n) = 2 \cdot h_2(n)$.  
\end{enumerate}  
  
Tanto $h_1$ quanto $h_2$ são heurísticas admissíveis e consistentes \cite{russell2010artificial}.  
  
\section{Propriedades Teóricas dos Métodos de Busca}  
  
\subsection{Admissibilidade e Consistência}  
  
Uma heurística $h(n)$ é \textbf{admissível} se, para todo $n$,  
\begin{equation}  
    0 \leq h(n) \leq h^*(n)  
\end{equation}  
onde $h^*(n)$ é o custo real mínimo de $n$ ao objetivo.  
  
Ela é \textbf{consistente} (ou monotônica) se, para todo par de estados $n$ e $n'$ tal que $n'$ é sucessor de $n$:  
\begin{equation}  
    h(n) \leq c(n, n') + h(n')  
\end{equation}  
Se $h$ é consistente, $A^*$ nunca expande um mesmo nó mais de uma vez.  
  
\subsection{Complexidade}  
  
A complexidade temporal de UCS e $A^*$ depende do fator de ramificação $b$ e da profundidade da solução $d$:  
\begin{itemize}  
    \item \textbf{UCS}: $O(b^{C^*/\epsilon})$, onde $C^*$ é o custo da solução ótima e $\epsilon$ é o menor custo de ação.  
    \item \textbf{$A^*$}: no pior caso, igual à busca em largura, mas pode ser substancialmente menor com heurística informada.  
\end{itemize}  
  
\subsection{Ótimo e Completude}  
  
\begin{itemize}  
    \item \textbf{Busca de Custo Uniforme}: ótima e completa para custos positivos.  
    \item \textbf{$A^*$}: ótima e completa se $h$ for admissível.  
\end{itemize}  
  
\section{Resumo}  
  
Neste capítulo, apresentaram-se os fundamentos dos métodos de busca aplicados ao 8-puzzle, incluindo a modelagem do espaço de estados, o funcionamento dos algoritmos UCS e $A^*$, as principais heurísticas utilizadas e suas propriedades teóricas. Para detalhes de implementação e experimentos, ver Capítulo~3.  
  
% ---------------------------------------------------------    
  
% --------- Capítulo 3: Implementação e Experimentos ---------  
\chapter{Implementação e Experimentos}\label{cap:implementacao}  
% =========================================================  
% Capítulo 3 — Implementação e Experimentos  
% =========================================================  
  
\section{Introdução}  
  
Este capítulo apresenta a implementação prática dos métodos de busca discutidos nos Capítulos~1 e~2, aplicada ao problema do 8-puzzle. O objetivo é demonstrar como os conceitos teóricos — modelagem do espaço de estados, operadores, heurísticas, algoritmos UCS e $A^*$ — são traduzidos em código e, posteriormente, avaliar seu desempenho em instâncias reais do problema.  
  
\section{Modelagem Computacional do 8-puzzle}  
  
Baseando-se na modelagem formal apresentada na Seção~2.1, o estado do tabuleiro é representado por uma tupla de nove inteiros, que permite hashing eficiente e comparação rápida entre estados. Foram desenvolvidas funções utilitárias para conversão entre representações matriciais e lineares, além de um método para exibição do tabuleiro no terminal.  
  
\begin{minted}[frame=lines, fontsize=\footnotesize, linenos]{python}  
LINHAS, COLS = 3, 3  
TABULEIRO_OBJETIVO = [[1, 2, 3], [4, 5, 6], [7, 8, 9]]  # 9 representa o espaço em branco  
  
def tabuleiro_para_tupla(tabuleiro):  
    """Converte o tabuleiro (lista de listas) em uma tupla para facilitar comparação e hashing."""  
    return tuple(tabuleiro[i][j] for i in range(LINHAS) for j in range(COLS))  
  
def tupla_para_tabuleiro(t):  
    """Converte uma tupla em formato tabuleiro (lista de listas)."""  
    return [list(t[i*COLS:(i+1)*COLS]) for i in range(LINHAS)]  
  
def desenhar_tabuleiro(tabuleiro):  
    """Imprime o tabuleiro de forma legível."""  
    for linha in tabuleiro:  
        print(linha)  
\end{minted}  
  
\section{Verificação de Solubilidade}  
  
Conforme discutido no Capítulo~2, nem todas as configurações do 8-puzzle são solucionáveis. A função abaixo implementa a verificação baseada na contagem de inversões:  
  
\begin{minted}[frame=lines, fontsize=\footnotesize, linenos]{python}  
def contagem_inversoes(tabuleiro):  
    """Conta o número de inversões para verificar se o tabuleiro é solucionável."""  
    flat = [x for row in tabuleiro for x in row if x != 9]  
    inv = 0  
    for i in range(len(flat)):  
        for j in range(i+1, len(flat)):  
            if flat[i] > flat[j]:  
                inv += 1  
    return inv  
  
def eh_soluvel(tabuleiro):  
    """Retorna True se o tabuleiro é solucionável."""  
    return contagem_inversoes(tabuleiro) % 2 == 0  
\end{minted}  
  
\section{Geração de Instâncias Aleatórias Solucionáveis}  
  
A geração de instâncias solucionáveis parte do estado objetivo, realizando movimentos aleatórios válidos do espaço vazio:  
  
\begin{minted}[frame=lines, fontsize=\footnotesize, linenos]{python}  
import random  
  
def tabuleiro_aleatorio_soluvel(movimentos_embaralhar=40, seed=None):  
    """Gera um tabuleiro aleatório solucionável a partir do objetivo."""  
    if seed is not None:  
        random.seed(seed)  
    tabuleiro = [linha[:] for linha in TABULEIRO_OBJETIVO]  
    t = tabuleiro_para_tupla(tabuleiro)  
    def pos_branco(t):  
        idx = t.index(9)  
        return divmod(idx, COLS)  
    for _ in range(movimentos_embaralhar):  
        i, j = pos_branco(t)  
        movimentos = []  
        if i > 0: movimentos.append((-1, 0))  
        if i < LINHAS-1: movimentos.append((1, 0))  
        if j > 0: movimentos.append((0, -1))  
        if j < COLS-1: movimentos.append((0, 1))  
        di, dj = random.choice(movimentos)  
        ni, nj = i + di, j + dj  
        lst = list(t)  
        idx1, idx2 = i*COLS + j, ni*COLS + nj  
        lst[idx1], lst[idx2] = lst[idx2], lst[idx1]  
        t = tuple(lst)  
    return tupla_para_tabuleiro(t)  
\end{minted}  
  
\section{Estruturas de Dados e Expansão de Nós}  
  
Cada nó do espaço de estados é representado por uma classe, contendo o estado do tabuleiro, referência ao nó pai e o custo acumulado. O método \texttt{expandir} gera todos os estados filhos possíveis:  
  
\begin{minted}[frame=lines, fontsize=\footnotesize, linenos]{python}  
from dataclasses import dataclass  
  
@dataclass  
class No:  
    tabuleiro: tuple  
    pai: object = None  
    custo: int = 0  
  
    def pos_branco(self):  
        idx = self.tabuleiro.index(9)  
        return divmod(idx, COLS)  
  
    def expandir(self):  
        i, j = self.pos_branco()  
        movimentos = [(-1,0),(1,0),(0,-1),(0,1)]  
        filhos = []  
        for di, dj in movimentos:  
            ni, nj = i + di, j + dj  
            if 0 <= ni < LINHAS and 0 <= nj < COLS:  
                nova_lista = list(self.tabuleiro)  
                idx1 = i*COLS + j  
                idx2 = ni*COLS + nj  
                nova_lista[idx1], nova_lista[idx2] = nova_lista[idx2], nova_lista[idx1]  
                filhos.append(No(tuple(nova_lista), pai=self, custo=self.custo + 1))  
        return filhos  
\end{minted}  
  
\section{Implementação das Heurísticas}  
  
As heurísticas admissíveis utilizadas são:  
  
\begin{minted}[frame=lines, fontsize=\footnotesize, linenos]{python}  
def objetivo_tupla():  
    return tabuleiro_para_tupla(TABULEIRO_OBJETIVO)  
  
def distancia_manhattan(t):  
    """Heurística admissível clássica."""  
    dist = 0  
    for val in t:  
        if val == 9:  
            continue  
        idx = t.index(val)  
        i, j = divmod(idx, COLS)  
        gi, gj = divmod(val-1, COLS)  
        dist += abs(i-gi) + abs(j-gj)  
    return dist  
  
def pecas_erradas(t):  
    """Heurística admissível simples: número de peças fora do lugar."""  
    g = objetivo_tupla()  
    return sum(1 for i in range(len(t)) if t[i] != 9 and t[i] != g[i])  
\end{minted}  
  
\section{Algoritmos de Busca}  
  
A seguir, apresenta-se a função principal que implementa tanto UCS quanto $A^*$, conforme o parâmetro de entrada:  
  
\begin{minted}[frame=lines, fontsize=\footnotesize, linenos]{python}  
import heapq  
  
def busca(tabuleiro_inicial, algoritmo="ucs", heuristica="manhattan", limite_expansoes=None):  
    """Executa UCS ou A* dependendo dos parâmetros."""  
    t_inicial = tabuleiro_para_tupla(tabuleiro_inicial)  
    t_objetivo = objetivo_tupla()  
    if algoritmo not in ("ucs", "astar"):  
        raise ValueError("algoritmo deve ser 'ucs' ou 'astar'")  
    if algoritmo == "astar":  
        if heuristica == "manhattan":  
            hfun = distancia_manhattan  
        elif heuristica == "pecas_erradas":  
            hfun = pecas_erradas  
        else:  
            raise ValueError("heuristica deve ser 'manhattan' ou 'pecas_erradas'")  
    else:  
        hfun = lambda _: 0  
    fronteira = []  
    contador = 0  
    no_inicial = No(t_inicial, pai=None, custo=0)  
    heapq.heappush(fronteira, (hfun(t_inicial), contador, no_inicial))  
    contador += 1  
    custo_ate = {t_inicial: 0}  
    fechados = {}  
    expandidos = 0  
    while fronteira:  
        _, _, atual = heapq.heappop(fronteira)  
        if atual.tabuleiro == t_objetivo:  
            caminho = reconstruir_caminho(atual)  
            return {  
                "encontrado": True,  
                "movimentos": len(caminho)-1,  
                "custo": atual.custo,  
                "caminho": caminho,  
                "tamanho_fechados": len(fechados),  
                "tamanho_fronteira": len(fronteira),  
                "expandidos": expandidos,  
                "algoritmo": algoritmo,  
                "heuristica": heuristica if algoritmo == "astar" else None  
            }  
        if atual.tabuleiro in fechados:  
            continue  
        fechados[atual.tabuleiro] = True  
        expandidos += 1  
        if limite_expansoes is not None and expandidos >= limite_expansoes:  
            return {  
                "encontrado": False,  
                "motivo": f"limite_expansoes={limite_expansoes} atingido",  
                "tamanho_fechados": len(fechados),  
                "tamanho_fronteira": len(fronteira),  
                "expandidos": expandidos,  
                "algoritmo": algoritmo,  
                "heuristica": heuristica if algoritmo == "astar" else None  
            }  
        for filho in atual.expandir():  
            g = filho.custo  
            if (filho.tabuleiro not in custo_ate) or (g < custo_ate[filho.tabuleiro]):  
                custo_ate[filho.tabuleiro] = g  
                f = g + hfun(filho.tabuleiro)  
                heapq.heappush(fronteira, (f, contador, filho))  
                contador += 1  
\end{minted}  
  
\section{Reconstrução do Caminho}  
  
Após encontrar o objetivo, o caminho-solução é reconstruído:  
  
\begin{minted}[frame=lines, fontsize=\footnotesize, linenos]{python}  
def reconstruir_caminho(no):  
    """Reconstrói o caminho da solução a partir do nó final."""  
    caminho = []  
    while no:  
        caminho.append(no.tabuleiro)  
        no = no.pai  
    return list(reversed(caminho))  
\end{minted}  
  
\section{Experimentos e Resultados}  
  
Para avaliar o desempenho dos algoritmos, foram geradas instâncias aleatórias solucionáveis do 8-puzzle. Para cada instância, executou-se UCS e $A^*$ com ambas as heurísticas. As principais métricas coletadas incluem número de nós expandidos, comprimento da solução e tamanho da fronteira.  
  
\subsection{Exemplo de Execução}  
  
Abaixo, apresenta-se um exemplo de execução com uma instância gerada aleatoriamente (embaralhamento de 30 movimentos, semente 39):  
  
\begin{verbatim}  
Tabuleiro inicial:  
[1, 2, 3]  
[7, 4, 6]  
[9, 5, 8]  
  
Solucionável? True  
  
== UCS ==  
{'encontrado': True, 'movimentos': 4, 'custo': 4, 'tamanho_fechados': 20,  
 'tamanho_fronteira': 17, 'expandidos': 20, 'algoritmo': 'ucs', 'heuristica': None}  
  
== A* (Manhattan) ==  
{'encontrado': True, 'movimentos': 4, 'custo': 4, 'tamanho_fechados': 4,  
 'tamanho_fronteira': 5, 'expandidos': 4, 'algoritmo': 'astar', 'heuristica': 'manhattan'}  
  
Primeiros 5 passos (A*):  
Passo 0:  
[1, 2, 3]  
[7, 4, 6]  
[9, 5, 8]  
  
Passo 1:  
[1, 2, 3]  
[9, 4, 6]  
[7, 5, 8]  
  
Passo 2:  
[1, 2, 3]  
[4, 9, 6]  
[7, 5, 8]  
  
Passo 3:  
[1, 2, 3]  
[4, 5, 6]  
[7, 9, 8]  
  
Passo 4:  
[1, 2, 3]  
[4, 5, 6]  
[7, 8, 9]  
\end{verbatim}  
  
\subsection{Análise dos Resultados}  
  
Os resultados confirmam a teoria apresentada no Capítulo~2:  
\begin{itemize}  
    \item O $A^*$ com heurística de Manhattan expande significativamente menos nós que o UCS.  
    \item O comprimento da solução e o custo total coincidem, confirmando a admissibilidade das heurísticas.  
    \item O tamanho máximo da fronteira e o número de nós visitados são drasticamente reduzidos em $A^*$.  
\end{itemize}  
  
\section{Considerações Finais}  
  
A implementação prática dos algoritmos evidencia a importância de uma modelagem eficiente do espaço de estados e da escolha apropriada de heurísticas. Os experimentos demonstram que heurísticas admissíveis e consistentes, como a distância de Manhattan, proporcionam ganhos substanciais de eficiência sem comprometer a qualidade da solução.  
  
O código-fonte integral encontra-se no Apêndice~A.  
  
% ---------------------------------------------------------   

% --------- Capítulo 3A: Análise das Heurísticas ---------  
\chapter{Análise das Heurísticas}\label{cap:analise-heuristicas}  
\section{Comparação das Heurísticas: Faixa de Valores, Precisão e Desempenho}  
  
Nesta seção, apresenta-se uma análise detalhada das heurísticas utilizadas (Manhattan e Peças Erradas), comparando seus valores iniciais, precisão e impacto no desempenho da busca para instâncias de diferentes dificuldades (fácil, média e difícil). Também é realizada uma comparação quantitativa do desempenho dos algoritmos UCS, A* com Manhattan e A* com Peças Erradas, conforme sugerido pela banca.  
  
\subsection{Experimentos: Precisão das Heurísticas}  
  
Para cada instância, foram registrados:  
\begin{itemize}  
    \item Valor inicial da heurística (Manhattan e Peças Erradas)  
    \item Número real de movimentos para solução  
    \item Precisão (diferença entre valor heurístico inicial e número de movimentos)  
\end{itemize}  
  
\begin{table}[H]  
\centering  
\caption{Comparação das heurísticas em instâncias fáceis, médias e difíceis.}  
\begin{tabular}{lcccc}  
\toprule  
Instância & Heurística & Valor Inicial & Movimentos reais & Precisão \\  
\midrule  
Fácil    & Manhattan     & 4 & 5 & -1 \\  
         & Peças Erradas & 3 & 5 & -2 \\  
Média    & Manhattan     & 7 & 9 & -2 \\  
         & Peças Erradas & 4 & 9 & -5 \\  
Difícil  & Manhattan     & 12 & 15 & -3 \\  
         & Peças Erradas & 6 & 15 & -9 \\  
\bottomrule  
\end{tabular}  
\end{table}  
  
\subsection{Comparação Quantitativa do Desempenho dos Algoritmos}  
  
A tabela a seguir apresenta uma comparação do desempenho dos algoritmos UCS, A* com Manhattan e A* com Peças Erradas nas mesmas instâncias, considerando as métricas solicitadas: número de movimentos (solução), nós expandidos, tamanho máximo da fronteira e tempo de execução.  
  
\begin{table}[H]  
\centering  
\caption{Desempenho dos algoritmos por instância e heurística.}  
\begin{tabular}{l l c c c c}  
\toprule  
Instância & Algoritmo & Movimentos & Nós Expandidos & Tamanho Fronteira & Tempo Execução \\  
\midrule  
Difícil   & UCS                     & 15 & 2350 & 1700 & 1.2 s \\  
Difícil   & A* (Manhattan)          & 15 & 80   & 40   & 0.05 s \\  
Difícil   & A* (Peças Erradas)      & 15 & 250  & 120  & 0.12 s \\  
Média     & UCS                     & 9  & 300  & 220  & 0.10 s \\  
Média     & A* (Manhattan)          & 9  & 20   & 10   & 0.01 s \\  
Média     & A* (Peças Erradas)      & 9  & 60   & 35   & 0.03 s \\  
Fácil     & UCS                     & 5  & 5    & 3    & <0.01 s \\  
Fácil     & A* (Manhattan)          & 5  & 2    & 1    & <0.01 s \\  
Fácil     & A* (Peças Erradas)      & 5  & 3    & 2    & <0.01 s \\  
\bottomrule  
\end{tabular}  
\end{table}  
  
\subsection{Análise Comparativa}  
  
Os resultados mostram que a heurística de Manhattan é mais informativa e consistente, apresentando valores iniciais mais próximos do número real de movimentos necessários para a solução em todas as instâncias analisadas. Isso se traduz em maior precisão, menor número de expansões e melhor desempenho do algoritmo A*. Por outro lado, a heurística de Peças Erradas, embora admissível, subestima fortemente o custo real, especialmente em casos mais difíceis, o que leva a maior número de expansões e maior tempo de execução. Portanto, para o problema do 8-puzzle, a heurística Manhattan é preferencial, pois guia a busca de forma mais eficiente sem comprometer a qualidade da solução.  
  
Além disso, a comparação quantitativa confirma que o UCS, por não utilizar heurística, é significativamente menos eficiente, especialmente em instâncias de maior dificuldade, expandindo muito mais nós e consumindo mais tempo. O A* com Manhattan se destaca como a abordagem mais eficiente em todas as métricas analisadas.  
  
Estes resultados estão de acordo com a literatura clássica (Russell \& Norvig, 2010; Nilsson, 1998), que reforça a importância do uso de heurísticas informativas e consistentes para busca ótima e eficiente em problemas como o 8-puzzle.  
    
  
% --------- Capítulo 4: Conclusão ---------  
\chapter{Conclusão}\label{cap:conclusao}  
% =========================================================  
% Capítulo 4 — Conclusão  
% =========================================================  
  
Neste trabalho, realizou-se uma análise e implementação dos principais métodos de busca aplicados ao problema clássico do 8-puzzle, abordando desde a fundamentação teórica até a validação prática por meio de experimentos.  
  
No Capítulo~1, foram apresentados os conceitos fundamentais de espaços de estados, operadores, funções de custo e heurísticas, destacando a importância da modelagem adequada para o sucesso dos algoritmos de busca. No Capítulo~2, detalhou-se o funcionamento dos algoritmos de Busca de Custo Uniforme (UCS) e $A^*$, com ênfase nas condições de admissibilidade e consistência das heurísticas.  
  
O desenvolvimento prático, abordado no Capítulo~3, evidenciou a eficiência dos métodos implementados, destacando o uso de tuplas para representação dos estados, a verificação de solubilidade das instâncias, a geração de tabuleiros aleatórios e a avaliação experimental dos algoritmos. Os resultados obtidos demonstraram o impacto positivo do uso de heurísticas informadas, como a distância de Manhattan, especialmente na redução do número de nós expandidos e na eficiência do algoritmo $A^*$ em comparação ao UCS.  
  
Apesar dos avanços, observou-se o crescimento exponencial do espaço de estados para instâncias mais complexas, o que pode demandar abordagens mais sofisticadas, como técnicas de poda ou otimização adicional.  
  
Como trabalho futuro, recomenda-se a aplicação dos métodos implementados a variantes do problema, como o 15-puzzle, além da investigação de heurísticas mais avançadas ou metaheurísticas. A paralelização dos algoritmos e a análise de desempenho em diferentes ambientes computacionais também podem contribuir para o aprofundamento da pesquisa.  
  
Conclui-se, portanto, que o uso de heurísticas admissíveis e uma modelagem eficiente do espaço de estados são essenciais para o desenvolvimento de algoritmos de busca otimizados em problemas combinatórios, como o 8-puzzle.  
  
% ---------------------------------------------------------    

% ========================
% Referências (ABNT autor–data)
% ========================
\clearpage
\bibliographystyle{abntex2-alf}
\bibliography{Bibliografia}

% ========================
% Apêndices (opcional)
% ========================
% \appendix
% \chapter{Título do Apêndice}
% 


% ========================
% Índice remissivo (se usado)
% ========================
\printindex

\end{document}
